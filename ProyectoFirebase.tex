\documentclass[10pt]{article} %% What type of document you're writing.

 
\usepackage{graphicx}

 
\usepackage{hyperref}

 
\usepackage[dvipsnames]{xcolor}

\usepackage[utf8]{inputenc}

 
%%%%% Preamble

%% Packages to use

 
\usepackage{amsmath,amsfonts,amssymb} %% AMS mathematics macros

%% Title Information.

 
\documentclass[10pt]{article} %% What type of document you're writing.

 
\usepackage{graphicx}

 
\usepackage{hyperref}

 
\usepackage[dvipsnames]{xcolor} 

 
\usepackage{amsmath,amsfonts,amssymb} %% AMS mathematics macros

 

%% Title Information.

 
\title{Aplicación web de peliculas (PELICULAS) \\ \\ \\ \\}

 
\author{Bernardo santos lopez  \\ \\ \\ \\ \\ Universidad Veracruzana.  \\ \\ \\ \\ \\ Facultad de Negocios y Tecnologías, campus Ixtac.  \\ \\ \\ \\ \\ Base de Dato No Convencionales. \\ \\ \\ \\ \\Catedrático: Doc. Adolfo Centeno Tellez \\ \\ \\ \\ \\
602 Ingeniría de Software. \\ \\ \\ \\ \\ }

 
%% \date{29 sep 2020} %% By default, LaTeX uses the current date

 
%%%%% The Document

 
\begin{document}
 
\maketitle


 
%\begin{abstract}
 
%This document implements the neural network, to find geometric \\ figures. \\ \\
 
%\end{abstract}


\section{Descripción}


Este proyecto tiene como objetivo desarrollar una aplicación web, con la cual se pueden subir las portadas de peliculas a la nube. Dicha página cuenta con un login en el cual se puede acceder a través de login con google o facebook. 

Para este proyecto se ocuparon las herramientas:
*Firebase
*Autentificación de firebase
*React js
*Bootstrap



 

 

 
\end{document}